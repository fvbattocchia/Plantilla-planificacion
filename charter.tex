\documentclass[11pt]{charter}

% El títulos de la memoria, se usa en la carátula y se puede usar el cualquier lugar del documento con el comando \ttitle
\titulo{Reproductor de audio digital con control de velocidad de reproducción} 

% Nombre del posgrado, se usa en la carátula y se puede usar el cualquier lugar del documento con el comando \degreename
\posgrado{Carrera de Especialización en Sistemas Embebidos} 
%\posgrado{Carrera de Especialización en Internet de las Cosas} 
%\posgrado{Carrera de Especialización en Intelegencia Artificial}
%\posgrado{Maestría en Sistemas Embebidos} 
%\posgrado{Maestría en Internet de las cosas}

% Tu nombre, se puede usar el cualquier lugar del documento con el comando \authorname
\autor{Florencia Battocchia} 

% El nombre del director y co-director, se puede usar el cualquier lugar del documento con el comando \supname y \cosupname y \pertesupname y \pertecosupname
\director{Pablo Slavkin}
\pertenenciaDirector{pertenencia} 
% FIXME:NO IMPLEMENTADO EL CODIRECTOR ni su pertenencia
\codirector{} % si queda vacio no se deberíá incluir 
\pertenenciaCoDirector{}

% Nombre del cliente, quien va a aprobar los resultados del proyecto, se puede usar con el comando \clientename y \empclientename
\cliente{Matias Battocchia}
\empresaCliente{Cliente particular}

% Nombre y pertenencia de los jurados, se pueden usar el cualquier lugar del documento con el comando \jurunoname, \jurdosname y \jurtresname y \perteunoname, \pertedosname y \pertetresname.
\juradoUno{Nombre y Apellido (1)}
\pertenenciaJurUno{pertenencia (1)} 
\juradoDos{Nombre y Apellido (2)}
\pertenenciaJurDos{pertenencia (2)}
\juradoTres{Nombre y Apellido (3)}
\pertenenciaJurTres{pertenencia (3)}
 
\fechaINICIO{22 de junio de 2020}		%Fecha de inicio de la cursada de GdP \fechaInicioName
\fechaFINALPlanificacion{22 de Agosto de 2020} 	%Fecha de final de cursada de GdP
\fechaFINALTrabajo{1 de junio de 2021}		%Fecha de defensa pública del trabajo final


\begin{document}

\maketitle
\thispagestyle{empty}
\pagebreak


\thispagestyle{empty}
{\setlength{\parskip}{0pt}
\tableofcontents{}
}
\pagebreak


\section{Registros de cambios}
\label{sec:registro}


\begin{table}[ht]
\label{tab:registro}
\centering

\begin{tabularx}{\linewidth}{@{}|c|X|c|@{}}
\hline
\rowcolor[HTML]{C0C0C0} 
Revisión & \multicolumn{1}{c|}{\cellcolor[HTML]{C0C0C0}Detalles de los cambios realizados} & Fecha      \\ \hline
1.0      & Creación del documento                                                          & 22/06/2020 \\ \hline
1.1      & Se modifica el título del documento \newline                                                                                Se modifica el Acta de constitución del proyecto \newline                                                                               
Se modifica la Descripción técnica-conceptual del proyecto a realizar \newline                                                                               
Se realiza la tabla de Identificación y análisis de los interesados \newline                                                                               
Se modifican las secciones 1,2,3 y 5 
						& 31/07/2020 \\ \hline
\end{tabularx}
\end{table}

\pagebreak



\section{Acta de constitución del proyecto}
\label{sec:acta}

\begin{flushright}
Buenos Aires, \fechaInicioName
\end{flushright}

\vspace{2cm}

Por medio de la presente se acuerda con la Ing. \authorname\hspace{1px} que su Trabajo Final de la \degreename\hspace{1px} se titulará ``\ttitle'', consistirá esencialmente en el prototipo preliminar de un dispositivo que almacena y reproduce un archivo de audio digital al cual se le podrá variar la velocidad de reproduccion del audio de forma permanente o temporal, y tendrá un presupuesto preliminar estimado de 600 hs de trabajo y \$242.775, con fecha de inicio \fechaInicioName\hspace{1px} y fecha de presentación pública \fechaFinalName.

Se adjunta a esta acta la planificación inicial.

\vfill

% Esta parte se construye sola con la información que hayan cargado en el preámbulo del documento y no debe modificarla
\begin{table}[ht]
\centering
\begin{tabular}{ccc}
\begin{tabular}[c]{@{}c@{}}Ariel Lutenberg \\ Director posgrado FIUBA\end{tabular} &  & \begin{tabular}[c]{@{}c@{}}\clientename \\ \empclientename \end{tabular} \vspace{2.5cm} \\ 
\multicolumn{3}{c}{\begin{tabular}[c]{@{}c@{}} \supname \\ Director del Trabajo Final\end{tabular}} \vspace{2.5cm} \\
\begin{tabular}[c]{@{}c@{}}\jurunoname \\ Jurado del Trabajo Final\end{tabular}     &  & \begin{tabular}[c]{@{}c@{}}\jurdosname\\ Jurado del Trabajo Final\end{tabular}  \vspace{2.5cm}  \\
\multicolumn{3}{c}{\begin{tabular}[c]{@{}c@{}} \jurtresname\\ Jurado del Trabajo Final\end{tabular}} \vspace{.5cm}                                                                     
\end{tabular}
\end{table}




\section{Descripción técnica-conceptual del proyecto a realizar}
\label{sec:descripcion}

\begin{consigna}{black}
Los primeros reproductores digitales para discjockeys aparecen a finales de los 80, su finalidad es poder reproducir la música almacenada en CDs de audio de manera similar a como se puede hacer con los platos giradiscos, es decir, pudiendo modificar la velocidad a la que se reproduce el tema para acompasarlo a otro ya sonando.

En 1993 comienza en este tipo de aparatos una fuerte revolución de la mano de Pioneer, que comercializa el CDJ-300, primer reproductor de la gama CDJ, gama que posteriormente ha dominado el mercado profesional de reproductores para DJ. Denon también desde el año 93 ha lanzado reproductores profesionales de gran aceptación, siendo la segunda marca en importancia por detrás de Pioneer en el mercado profesional.

En 2001 Pioneer volvió a revolucionar el mercado con el CDJ-1000, un reproductor de CDs capaz de emular más que cualquier otro producto lanzado hasta el momento el funcionamiento de un plato giradiscos gracias a que con su jog de gran tamaño se podían emular las técnicas de scratch que se realizan con platos y vinilos. 

A partir del año 2002 comienzan a surgir reproductores profesionales con capacidad para reproducir archivos MP3. Desde el año 2007 la capacidad de reproducir archivos MP3 comienza a convertirse en algo habitual en casi todos los reproductores, que ya son capaces de leer este tipo de archivos desde discos CD-R, pendrives USB o tarjetas de memoria. En la actualidad muchos dispositivos son capaces también de leer otros formatos como el M4A o el AAC, así como formatos sin compresión como WAV o AIFF.

En los últimos 5 años se han popularizado también los reproductores que pueden ser empleados como controladores MIDI con software para Djs.

Si bien en el mercado existe una gran variedad de estos equipos y con múltiples funciones(\textit{display, play/pause,CUE, track search, autocue,pitch control, pitch bend, master tempo, tempo range, jogwheel, loop, hot cue, navegación, efectos, slip}) son importados y extremadamente caros por ejemplo se muestra en la figura \ref{fig:Pioneer} un equipo Pioneer Cdj-2000 Nexus, que cuesta \$454.400

\begin{figure}[htpb]
\centering 
\includegraphics[width=.7\textwidth]{./Figuras/reproductordigital.jpeg}
\caption{Pioneer Cdj-2000 Nexus}
\label{fig:Pioneer}
\end{figure}

El presente proyecto se destaca especialmente por desarrollar una alternativa económica y que mantenga la funcionalidad simple del vinilo pero en forma digital, implementando las funcionalidades de \textit{play/pause, pitch control, pitch Bend}. El control \textit{play/pause} se realiza con un botón y su función es iniciar la reproducción o ponerla en pausa. El \textit{pitch control} se lleva a cabo con un potenciómetro deslizante y su función es la de modificar la velocidad de reproducción de la canción, posicionado en la parte central de su recorrido, la velocidad de reproducción no es afectada, deslizándolo hacia abajo aumenta y hacia arriba disminuye la velocidad de reproducción del audio. El control \textit{ pitch Bend} se realiza con dos botones marcados como +  y - , mientras se pulsan la velocidad de reproducción varía momentaneamente, uno la aumenta y el otro la disminuye.

Un diagrama general del trabajo se presenta en la figura \ref{fig:diagBloques}. El procesador de audio digital se encuentra representado por el bloque amarillo, mientras que en los bloques color azul se encuentra representado el ambiente con el cual interactúa el procesador. En la entrada del procesador se encuentra la consola que contiene la interfaz con el usuario y la unidad de almacenamiento que contiene el audio en formato wav PCM (\textit{Pulse-Code Modulation}). El procesador lee el audio almacenado, lo procesa según los eventos que recibe de la consola y lo convierte de PCM a PWM. La señal de  salida se amplifica por medio de FETs y se le aplica un filtro de reconstrucción. La salida de audio presenta formato RCA y se conecta a un equipo con entrada line-in que permite escuchar la canción como por ejemplo un \textit{mixer} (mezcladora), este se encuentra en el recuadro verde y no pertene al desarrollo del proyecto. 

\begin{figure}[htpb]
\centering 
\includegraphics[width=.9\textwidth]{./Figuras/diagBloquesDSP.png}
\caption{Diagrama general del equipo.}
\label{fig:diagBloques}
\end{figure}

En la seguiente figura \ref{fig:consola} se detalla la interfaz con el usurio, donde el potenciometro lineal varía la velocidad de reproducción del audio de forma permanente (\textit{pitch control}), aumentando la velocidad cuando se de desplaza hacia abajo, produciendo un sonido mas agudo y disminuyendo la velocidad cuando se lo desplaza hacia arriba produciendo un sonido mas grave. Los botonos con símbolos menos (-) y mas (+) cambian la velocidad de reproducción del audio de forma temportal mientra se los mantiene pulsados (\textit{pitch bend}), también se puede observar en esta figura el botón de play/pausa.

\begin{figure}[htpb]
\centering 
\includegraphics[width=.5\textwidth]{./Figuras/interfaz.png}
\caption{Interfaz de usuario.}
\label{fig:consola}
\end{figure}

\end{consigna}
\section{Identificación y análisis de los interesados}
\label{sec:interesados}

\begin{consigna}{black} 

\begin{table}[ht]
%\caption{Identificación de los interesados}
%\label{tab:interesados}
\begin{tabularx}{\linewidth}{@{}|l|X|X|l|@{}}
\hline
\rowcolor[HTML]{C0C0C0} 
Rol           & Nombre y Apellido & Organización 	& Puesto 	\\ \hline
Cliente       & \clientename      &\empclientename	&        	\\ \hline
Responsable   & \authorname       & FIUBA        	& Alumno 	\\ \hline
Orientador    & \supname	      & \pertesupname 	& Director	Trabajo final \\ \hline
Usuario final & \clientename      & \empclientename	&        	\\ \hline
\end{tabularx}
\end{table}


\end{consigna}



\section{1. Propósito del proyecto}
\label{sec:proposito}

\begin{consigna}{black}
El propósito de este proyecto es desarrollar un reproductor de audio digital con control de velocidad de reproducción para obtener un dispositivo de características similares a las del vinilo en forma digital a un costo inferior a los existentes en el mercado. 
\end{consigna}

\section{2. Alcance del proyecto}
\label{sec:alcance}

\begin{consigna}{black}
El siguiente proyecto incluye:
\begin{itemize}
\item Soporte para almacenamiento de audio.
\item Control de velocidad de la reproducción del audio de forma temporal.
\item Control de velocidad de la reproducción del audio de forma permanente.
\item Reproducción y pausa del audio.
\item Salida de audio por dos canales.
\end{itemize}

El presente proyecto no incluye:
\begin{itemize}
\item Formatos de audio comprimido MP3, AAC, FLAC, ALAC.
\item Pantalla LCD con información de la pista, tiempo transcurrido, tiempo restante, valor de ajuste de velocidad, forma de onda.
\item Controles adicionales para seleccionar y cargar pistas.
\end{itemize}
\end{consigna}

\section{3. Requerimientos}
\label{sec:requerimientos}
\begin{consigna}{black}
\begin{enumerate}
\item Requerimientos del sistema operativo:
	\begin{enumerate}
	\item Se deberá utilizar un sistema operativo de tiempo real.
	\end{enumerate}
\item Requerimientos de Hardware:
	\begin{enumerate}
	\item  Se deberá utilizará la CIAA-NXP como computadora principal.
	\end{enumerate}
\item Requerimientos de interfaces externas:
	\begin{enumerate}
	\item Deberá tener un botón de reproducción/pausa del audio.
	\item Deberá tener 2 botones de ajuste temporal de velocidad que disminuyen o incrementan la velocidad mientras se encuentran pulsados.
	\item Deberá tener un potenciómetro lineal para el ajuste permanente de velocidad.
	\item Deberá utilizar un conector RCA estéreo para por conectar el dispositivo a un equipo con entrada  line-in.
	\end{enumerate}
\item Requerimientos de fuente de audio:
	\begin{enumerate}
	\item Se deberá almacenar el archivo de audio en en una memoria SD
	\item El archivo de audio deberá tener formato .wav con 44100 Hz, calidad estéreo de 16 bits.
	\item El archivo de audio deberá tener formato PCM.
	\end{enumerate}
\item Requerimientos de salida de sonido:
	\begin{enumerate}
	\item La velocidad normar de reproducción del audio deberó ser de 44100 Hz.
	\item El \textit{pith control} deberá aumentar o disminuir la velocidad normal de reproducción hasta un un 8\%.
	\item El \textit{pinch bend} deberá aumentar o disminuir la velocidad normal de reproducción un 5\%.
	\item Se deberá implementar dos canales de salida PWM.
	\item El nivel de salida deberá ser de 2,0 Vrms.
	\end{enumerate}
\item Requerimientos de alimentacion:
	\begin{enumerate}
	\item La alimentación deberá ser de 5 VCC por medio de USB.
	\end{enumerate}
\end{enumerate}
\end{consigna}

\section{4. Entregables principales del proyecto}
\label{sec:entregables}

\begin{consigna}{black}
\begin{itemize}
\item Manual de uso
\item Código fuente
\item Esquemático de conexionado.
\item Informe final

\end{itemize}

\end{consigna}

\section{5. Desglose del trabajo en tareas}
\label{sec:wbs}

\begin{consigna}{black}

\begin{enumerate}
\item Planificación del proyecto (10hs)
	\begin{enumerate}
	\item Realizar la planificación del proyecto (10hs)
	\end{enumerate}
\item Recopilación general de información sobre el proyecto (70 hs)
	\begin{enumerate}
	\item Realizar el análisis y elección de la electrónica y el microcontrolador a utilizar (10hs)
	\item Investigar sobre algoritmos de procesamiento de señales digitales en tiempo real (30hs)
	\item Investigar sobre técnicas de conversión de señal digital a analógica : DAC, PWM o r2r network 		dac (30hs)
	\end{enumerate}
\item Diseño de hardware (20 hs)
	\begin{enumerate}
	\item Diseño esquemático de conexionado. (20hs)
	\end{enumerate}
\item Implementación de la consola (50hs)
	\begin{enumerate}
	\item Realizar el circuito electrónico de la interfaz de usuario (15hs)
	\item Testeo de del circuito electrónico. (5hs)
	\item Realizar la lectura de las entradas de la interfaz de usuario. (30hs)
	\end{enumerate}
\item Adquisiscion de audio de la memoria SD (32hs)
	\begin{enumerate}
	\item Realizar circuito electrónico para conectar la memoria SD. (5hs)
	\item Testeo del circuito electrónico. (5hs)
	\item Burcar audio que cumpla con el requisito 4. y almacenarlo en la memoria SD. (2hs)
	\item Realizar la lectura del audio almacenado en la tarjeta SD. (20hs)
	\end{enumerate}
\item Implementació del sistema operativo (40hs)
	\begin{enumerate}
	\item Implementar sistema operativo RTEMS (20hs)
	\item Implementar manejo de las interrupciones provenientes de la interfaz de usuario (20hs) 
	\end{enumerate}
\item Implementación de los algoritmos de procesamiento de señales (165hs)
	\begin{enumerate}
	\item Implementar algoritmo de re-muestreo para aumentar y disminuir la velocidad de reproducción del audio. (40hs)
	\item Implementar algoritmo de interpolación.(40hs)
	\item Implementar algoritmo para reducir la profundiad de bits del audio.(40hs)
	\item Implementar algoritmo para convertir el audio de PCM a PWM por dos canales (40hs)
	\item Testeo de la salida PWM con osciloscopio y analizador de espectros. (5 hs)
	\end{enumerate}
\item Salida de audio (23 hs)
	\begin{enumerate}	
	\item Realizar circuito electrónico del amplificador en protoboard. (5hs)
	\item Realizar circuito electrónico del filtro de reconstrucción en protoboard. (5hs)
	\item Realizar circuito electrónico del conector RCA en protoboard. (5hs)
	\item Testeo de los circuitos electrónico. (8hs)
	\end{enumerate}
\item Integración del sistema (30hs)
	\begin{enumerate}
	\item Integración de los módulos del sistema (20hs).
	\item Testeo de funcionamiento (10hs).
	\end{enumerate}
\item Pruebas (80hs)
	\begin{enumerate}
	\item Desarrollar herramientas de testeo, debug y validación. (40 hs)
	\item Realizar las pruebas de validación del sistema (40 hs)
	\end{enumerate}
\item Documentación (80 hs)
	\begin{enumerate}
	\item Realizar el informe final del proyecto (50hs)
	\item Realizar el manual de uso (30hs)
	\end{enumerate}
\end{enumerate}

Cantidad total de horas: (600 hs)

\end{consigna}

\section{6. Diagrama de Activity On Node}
\label{sec:AoN}

\begin{consigna}{black}
En la siguiente figura \ref{fig:AoN} observa el diagrama de Activity On Node.

\begin{figure}[htpb]
\centering 
\includegraphics[width=.9\textwidth]{./Figuras/AoN.png}
\caption{Diagrama en \textit{Activity on Node}}
\label{fig:AoN}
\end{figure}


Los tiempos expresados en el diagrama se encuentran en horas. El color de cada recuadro
representa el grupo de tareas contenidas en el punto.

\end{consigna}



\section{7. Diagrama de Gantt}
\label{sec:gantt}
\begin{consigna}{black}

A continuación observa el diagrama de Gantt \ref{fig:gant1}

\begin{figure}[htpb]
\centering 
\includegraphics[width=1.1\textwidth]{./Figuras/gant1.png}
\caption{Diagrama de \textit{Gantt}}
\label{fig:gant1}
\end{figure}

\begin{figure}[htpb]
\centering 
\includegraphics[width=1.1\textwidth]{./Figuras/gant2.png}
\caption{Diagrama de \textit{Gantt continuación}}
\label{fig:gant1}
\end{figure}

\begin{figure}[htpb]
\centering 
\includegraphics[width=1.1\textwidth]{./Figuras/gant3.png}
\caption{Diagrama de \textit{Gantt continuación}}
\label{fig:gant1}
\end{figure}

\begin{figure}[htpb]
\centering 
\includegraphics[width=1.1\textwidth]{./Figuras/gant4.png}
\caption{Diagrama de \textit{Gantt continuación}}
\label{fig:gant1}
\end{figure}

\end{consigna}

\section{8. Matriz de uso de recursos de materiales}
\label{sec:recursos}
\begin{consigna}{black}

\begin{table}[]
\begin{tabular}{|c|c|c|c|c|c|}
\hline
\rowcolor[HTML]{C0C0C0} 
\cellcolor[HTML]{C0C0C0}                                                                       & \cellcolor[HTML]{C0C0C0}                                                           & \multicolumn{4}{c|}{\cellcolor[HTML]{C0C0C0}Recursos Requeridos (Horas)}                               \\ \cline{3-6} 
\rowcolor[HTML]{C0C0C0} 
\multirow{-2}{*}{\cellcolor[HTML]{C0C0C0}\begin{tabular}[c]{@{}c@{}}Código\\ WBS\end{tabular}} & \multirow{-2}{*}{\cellcolor[HTML]{C0C0C0}Nombre de la tarea}                       & PC & EDU-CIAA-NXP & Osciloscopio & \begin{tabular}[c]{@{}c@{}}Analizador\\ de\\ espectros\end{tabular} \\ \hline
\rowcolor[HTML]{CBCEFB} 
1                                                                                              & Planificación del proyecto                                                         &    &              &              &                                                                     \\ \hline
1.1                                                                                            & Realizar la planificación del proyecto                                             & 10 &              &              &                                                                     \\ \hline
\rowcolor[HTML]{CBCEFB} 
2                                                                                              & Recopilación general de información sobre el proyecto                              &    &              &              &                                                                     \\ \hline
2.1                                                                                            & Realizar el análisis y elección de la electrónica y el microcontrolador a utilizar & 10 &              &              &                                                                     \\ \hline
2.2                                                                                            & Investigar sobre algoritmos de procesamiento de señales digitales en tiempo real   & 30 &              &              &                                                                     \\ \hline
2.3                                                                                            & Investigar sobre técnicas de conversión de señal digital a analógica               & 30 &              &              &                                                                     \\ \hline
\rowcolor[HTML]{CBCEFB} 
3                                                                                              & Diseño de hardware                                                                 &    &              &              &                                                                     \\ \hline
3.1                                                                                            & Diseño esquemático de conexionado.                                                 & 20 &              &              &                                                                     \\ \hline
\rowcolor[HTML]{CBCEFB} 
4                                                                                              & Implementació de la consola                                                        &    &              &              &                                                                     \\ \hline
4.1                                                                                            & Realizar el circuito electrónico de la interfaz de usuario                         &    &              &              &                                                                     \\ \hline
4.2                                                                                            & Testeo de del circuito electrónico                                                 &    &              &              &                                                                     \\ \hline
4.3                                                                                            & Realizar la lectura de las entradas de la interfaz de usuario                      & 30 & 30           &              &                                                                     \\ \hline
\rowcolor[HTML]{CBCEFB} 
5                                                                                              & Adquisiscion de audio de la memoria SD                                             &    &              &              &                                                                     \\ \hline
5.1                                                                                            & Realizar circuito electrónico para conectar la memoria SD. (5hs)                   &    &              &              &                                                                     \\ \hline
5.2                                                                                            & Testeo del circuito electrónico.                                                   &    &              &              &                                                                     \\ \hline
5.3                                                                                            & Burcar audio que cumpla con el requerimiento  y almacenarlo en la memoria SD.      & 2  &              &              &                                                                     \\ \hline
5.4                                                                                            & Realizar la lectura del audio almacenado en la tarjeta SD.                         & 20 & 20           &              &                                                                     \\ \hline
\rowcolor[HTML]{CBCEFB} 
6                                                                                              & Implementació del sistema operativo                                                &    &              &              &                                                                     \\ \hline
6.1                                                                                            & Implementar sistema operativo RTEMS                                                & 20 & 20           &              &                                                                     \\ \hline
6.2                                                                                            & Implementar manejo de las interrupciones  provenientes de la interfaz de usuario   & 20 & 20           &              &                                                                     \\ \hline
\rowcolor[HTML]{CBCEFB} 
7                                                                                              & Implementación de los algoritmos de procesamiento de señales                       &    &              &              &                                                                     \\ \hline
7.1                                                                                            & Implementar algoritmo de re-muestreo para controlar la velocidad de reproducción   & 40 & 40           &              &                                                                     \\ \hline
7.2                                                                                            & Implementar algoritmo de interpolación.                                            & 40 & 40           &              &                                                                     \\ \hline
7.3                                                                                            & Implemetar algoritmo para reducir la profundiad de bits                            & 40 & 40           &              &                                                                     \\ \hline
7.4                                                                                            & Implementar algoritmo para convertir el audio de PCM a PWM por dos canales         & 40 & 40           &              &                                                                     \\ \hline
7.5                                                                                            & Testeo de la salida PWM con osciloscopio y analizador de espectros                 &    &              & 2            & 3                                                                   \\ \hline
\rowcolor[HTML]{CBCEFB} 
8                                                                                              & Salida de audio                                                                    &    &              &              &                                                                     \\ \hline
8.1                                                                                            & Realizar circuito electrónico del amplificador en protoboard                       &    &              &              &                                                                     \\ \hline
8.2                                                                                            & Realizar circuito electrónico del filtro de reconstrucción en protoboard           &    &              &              &                                                                     \\ \hline
8.3                                                                                            & Realizar circuito electrónico del conector RCA en protoboard                       &    &              &              &                                                                     \\ \hline
8.4                                                                                            & Testeo de los circuitos electrónico                                                &    &              & 4            &                                                                     \\ \hline
\rowcolor[HTML]{CBCEFB} 
9                                                                                              & Integración del sistema                                                            &    &              &              &                                                                     \\ \hline
9.1                                                                                            & Integración de los módulos del sistema                                             & 20 & 20           &              &                                                                     \\ \hline
9.2                                                                                            & Testeo de funcionamiento                                                           &    &              & 5            & 5                                                                   \\ \hline
\rowcolor[HTML]{CBCEFB} 
10                                                                                             & Pruebas                                                                            &    &              &              &                                                                     \\ \hline
10.1                                                                                           & Desarrollar herramientas de testeo, debug y validación.                            & 40 &              &              &                                                                     \\ \hline
10.2                                                                                           & Realizar las pruebas de validación del sistema                                     & 40 & 40           &              &                                                                     \\ \hline
\rowcolor[HTML]{CBCEFB} 
11                                                                                             & Documentación                                                                      &    &              &              &                                                                     \\ \hline
11.1                                                                                           & Realizar el informe final del proyecto                                             & 50 &              &              &                                                                     \\ \hline
11.2                                                                                           & Realizar el manual de uso                                                          & 30 &              &              &                                                                     \\ \hline
\end{tabular}
\end{table}
\end{consigna}
\section{9. Presupuesto detallado del proyecto}
\label{sec:presupuesto}
\begin{table}[]
\begin{tabular}{|l|c|c|c|}
\hline
\rowcolor[HTML]{C0C0C0} 
\multicolumn{4}{|c|}{\cellcolor[HTML]{C0C0C0}COSTOS DIRECTOS}                                                                \\ \hline
\rowcolor[HTML]{C0C0C0} 
\multicolumn{1}{|c|}{\cellcolor[HTML]{C0C0C0}Descripción} & Cantidad              & Valor unitario (\$)   & Valor total (\$) \\ \hline
EDU-CIAA-NXP                                              & 1                     & 4600                  & 4600             \\ \hline
memoria SD                                                & 1                     & 550                   & 550              \\ \hline
potenciómetro lineal                                      & 1                     & 220                   & 200              \\ \hline
transistor FET                                            & 2                     & 200                   & 400              \\ \hline
cable RCA                                                 & 1                     & 1000                  & 1000             \\ \hline
componentes varios                                        & \multicolumn{1}{l|}{} & 1000                  & 1000             \\ \hline
Hs/hombre                                                 & 600Hs                 & 300                   & 180000           \\ \hline
\multicolumn{3}{|c|}{SUBTOTAL}                                                                            & 186750           \\ \hline
\rowcolor[HTML]{C0C0C0} 
\multicolumn{4}{|c|}{\cellcolor[HTML]{C0C0C0}COSTOS INDIRECTOS}                                                              \\ \hline
\rowcolor[HTML]{C0C0C0} 
\multicolumn{1}{|c|}{\cellcolor[HTML]{C0C0C0}Descripción} & Cantidad              & Valor unitario        & Valor total      \\ \hline
30\% de los costos directos                               & \multicolumn{1}{l|}{} & \multicolumn{1}{l|}{} & 56025            \\ \hline
\multicolumn{3}{|c|}{SUBTOTAL}                                                                            &                  \\ \hline
\rowcolor[HTML]{C0C0C0} 
\multicolumn{3}{|c|}{\cellcolor[HTML]{C0C0C0}TOTAL}                                                       & 242775           \\ \hline
\end{tabular}
\end{table}


\section{10. Matriz de asignación de responsabilidades}
\label{sec:responsabilidades}

\begin{table}[]
\begin{tabular}{|c|c|c|c|c|}
\hline
\rowcolor[HTML]{C0C0C0} 
\cellcolor[HTML]{C0C0C0}                                                                       & \cellcolor[HTML]{C0C0C0}                                                           & \multicolumn{3}{c|}{\cellcolor[HTML]{C0C0C0}Nombres y roles definidos en el proyecto}                                                                                                                                   \\ \cline{3-5} 
\rowcolor[HTML]{C0C0C0} 
\multirow{-2}{*}{\cellcolor[HTML]{C0C0C0}\begin{tabular}[c]{@{}c@{}}Código\\ WBS\end{tabular}} & \multirow{-2}{*}{\cellcolor[HTML]{C0C0C0}Nombre de la tarea}                       & \begin{tabular}[c]{@{}c@{}}Responsable\\  Florencia Battocchia\end{tabular} & \begin{tabular}[c]{@{}c@{}}Orientador\\ Pablo Slavkin\end{tabular} & \begin{tabular}[c]{@{}c@{}}Cliente\\  Matias Battocchia\end{tabular} \\ \hline
\rowcolor[HTML]{CBCEFB} 
1                                                                                              & Planificación del proyecto                                                         &                                                                             &                                                                    &                                                                      \\ \hline
1.1                                                                                            & Realizar la planificación del proyecto                                             & P                                                                           & A                                                                  & A                                                                    \\ \hline
\rowcolor[HTML]{CBCEFB} 
2                                                                                              & Recopilación general de información sobre el proyecto                              &                                                                             &                                                                    &                                                                      \\ \hline
2.1                                                                                            & Realizar el análisis y elección de la electrónica y el microcontrolador a utilizar & P                                                                           & I                                                                  & C                                                                    \\ \hline
2.2                                                                                            & Investigar sobre algoritmos de procesamiento de señales digitales en tiempo real   & P                                                                           & I                                                                  & I                                                                    \\ \hline
2.3                                                                                            & Investigar sobre técnicas de conversión de señal digital a analógica               & P                                                                           & I                                                                  & I                                                                    \\ \hline
\rowcolor[HTML]{CBCEFB} 
3                                                                                              & Diseño de hardware                                                                 &                                                                             &                                                                    &                                                                      \\ \hline
3.1                                                                                            & Diseño esquemático de conexionado.                                                 & P                                                                           & A                                                                  & I                                                                    \\ \hline
\rowcolor[HTML]{CBCEFB} 
4                                                                                              & Implementació de la consola                                                        &                                                                             &                                                                    &                                                                      \\ \hline
4.1                                                                                            & Realizar el circuito electrónico de la interfaz de usuario                         & P                                                                           & I                                                                  & I                                                                    \\ \hline
4.2                                                                                            & Testeo de del circuito electrónico                                                 & P                                                                           & I                                                                  & I                                                                    \\ \hline
4.3                                                                                            & Realizar la lectura de las entradas de la interfaz de usuario                      & P                                                                           & I                                                                  & I                                                                    \\ \hline
\rowcolor[HTML]{CBCEFB} 
5                                                                                              & Adquisiscion de audio de la memoria SD                                             &                                                                             &                                                                    &                                                                      \\ \hline
5.1                                                                                            & Realizar circuito electrónico para conectar la memoria SD. (5hs)                   & P                                                                           & I                                                                  & I                                                                    \\ \hline
5.2                                                                                            & Testeo del circuito electrónico.                                                   & P                                                                           & I                                                                  & I                                                                    \\ \hline
5.3                                                                                            & Burcar audio que cumpla con el requerimiento  y almacenarlo en la memoria SD.      & P                                                                           & I                                                                  & C                                                                    \\ \hline
5.4                                                                                            & Realizar la lectura del audio almacenado en la tarjeta SD.                         & P                                                                           & I                                                                  & I                                                                    \\ \hline
\rowcolor[HTML]{CBCEFB} 
6                                                                                              & Implementació del sistema operativo                                                &                                                                             &                                                                    &                                                                      \\ \hline
6.1                                                                                            & Implementar sistema operativo RTEMS                                                & P                                                                           & I                                                                  & I                                                                    \\ \hline
6.2                                                                                            & Implementar manejo de las interrupciones  provenientes de la interfaz de usuario   & P                                                                           & I                                                                  & I                                                                    \\ \hline
\rowcolor[HTML]{CBCEFB} 
7                                                                                              & Implementación de los algoritmos de procesamiento de señales                       &                                                                             &                                                                    &                                                                      \\ \hline
7.1                                                                                            & Implementar algoritmo de re-muestreo para controlar la velocidad de reproducción   & P                                                                           & C                                                                  & I                                                                    \\ \hline
7.2                                                                                            & Implementar algoritmo de interpolación.                                            & P                                                                           & C                                                                  & I                                                                    \\ \hline
7.3                                                                                            & Implemetar algoritmo para reducir la profundiad de bits                            & P                                                                           & C                                                                  & I                                                                    \\ \hline
7.4                                                                                            & Implementar algoritmo para convertir el audio de PCM a PWM por dos canales         & P                                                                           & C                                                                  & I                                                                    \\ \hline
7.5                                                                                            & Testeo de la salida PWM con osciloscopio y analizador de espectros                 & P                                                                           & A                                                                  & I                                                                    \\ \hline
\rowcolor[HTML]{CBCEFB} 
8                                                                                              & Salida de audio                                                                    &                                                                             &                                                                    &                                                                      \\ \hline
8.1                                                                                            & Realizar circuito electrónico del amplificador en protoboard                       & P                                                                           & I                                                                  & I                                                                    \\ \hline
8.2                                                                                            & Realizar circuito electrónico del filtro de reconstrucción en protoboard           & P                                                                           & I                                                                  & I                                                                    \\ \hline
8.3                                                                                            & Realizar circuito electrónico del conector RCA en protoboard                       & P                                                                           & I                                                                  & I                                                                    \\ \hline
8.4                                                                                            & Testeo de los circuitos electrónico                                                & P                                                                           & I                                                                  & I                                                                    \\ \hline
\rowcolor[HTML]{CBCEFB} 
9                                                                                              & Integración del sistema                                                            &                                                                             &                                                                    &                                                                      \\ \hline
9.1                                                                                            & Integración de los módulos del sistema                                             & P                                                                           & I                                                                  & I                                                                    \\ \hline
9.2                                                                                            & Testeo de funcionamiento                                                           & P                                                                           & A                                                                  & I                                                                    \\ \hline
\rowcolor[HTML]{CBCEFB} 
10                                                                                             & Pruebas                                                                            &                                                                             &                                                                    &                                                                      \\ \hline
10.1                                                                                           & Desarrollar herramientas de testeo, debug y validación.                            & P                                                                           & A                                                                  & A                                                                    \\ \hline
10.2                                                                                           & Realizar las pruebas de validación del sistema                                     & P                                                                           & A                                                                  & A                                                                    \\ \hline
\rowcolor[HTML]{CBCEFB} 
11                                                                                             & Documentación                                                                      &                                                                             &                                                                    &                                                                      \\ \hline
11.1                                                                                           & Realizar el informe final del proyecto                                             & P                                                                           & A                                                                  & A                                                                    \\ \hline
11.2                                                                                           & Realizar el manual de uso                                                          & P                                                                           & A                                                                  & A                                                                    \\ \hline
\end{tabular}
\end{table}
{\footnotesize
Referencias:
\begin{itemize}
	\item P = Responsabilidad Primaria
	\item S = Responsabilidad Secundaria
	\item A = Aprobación
	\item I = Informado
	\item C = Consultado
\end{itemize}
} %footnotesize

\section{11. Gestión de riesgos}
\label{sec:riesgos}

\begin{consigna}{red}
a) Identificación de los riesgos (al menos cinco) y estimación de sus consecuencias:
 
Riesgo 1: detallar el riesgo (riesgo es algo que si ocurre altera los planes previstos)
\begin{itemize}
\item Severidad (S): mientras más severo, más alto es el número (usar números del 1 al 10).\\
Justificar el motivo por el cual se asigna determinado número de severidad (S).
\item Probabilidad de ocurrencia (O): mientras más probable, más alto es el número (usar del 1 al 10).\\
Justificar el motivo por el cual se asigna determinado número de (O). 
\end{itemize}   

Riesgo 2:
\begin{itemize}
\item Severidad (S): 
\item Ocurrencia (O):
\end{itemize}

Riesgo 3:
\begin{itemize}
\item Severidad (S): 
\item Ocurrencia (O):
\end{itemize}


b) Tabla de gestión de riesgos:      (El RPN se calcula como RPN=SxO)

\begin{table}[htpb]
\centering
\begin{tabularx}{\linewidth}{@{}|X|c|c|c|c|c|c|@{}}
\hline
\rowcolor[HTML]{C0C0C0} 
Riesgo & S & O & RPN & S* & O* & RPN* \\ \hline
       &   &   &     &    &    &      \\ \hline
       &   &   &     &    &    &      \\ \hline
       &   &   &     &    &    &      \\ \hline
       &   &   &     &    &    &      \\ \hline
       &   &   &     &    &    &      \\ \hline
\end{tabularx}%
\end{table}

Criterio adoptado: 
Se tomarán medidas de mitigación en los riesgos cuyos números de RPN sean mayores a ....

Nota: los valores marcados con (*) en la tabla corresponden luego de haber aplicado la mitigación.

c) Plan de mitigación de los riesgos que originalmente excedían el RPN máximo establecido:
 
Riesgo 1: Plan de mitigación (si por el RPN fuera necesario elaborar un plan de mitigación).
  Nueva asignación de S y O, con su respectiva justificación:
  - Severidad (S): mientras más severo, más alto es el número (usar números del 1 al 10).
          Justificar el motivo por el cual se asigna determinado número de severidad (S).
  - Probabilidad de ocurrencia (O): mientras más probable, más alto es el número (usar del 1 al 10).
          Justificar el motivo por el cual se asigna determinado número de (O).

Riesgo 2: Plan de mitigación (si por el RPN fuera necesario elaborar un plan de mitigación).
 
Riesgo 3: Plan de mitigación (si por el RPN fuera necesario elaborar un plan de mitigación)

\end{consigna}


\section{13. Gestión de la calidad}
\label{sec:calidad}

\begin{consigna}{red}
Para cada uno de los requerimientos del proyecto indique:
\begin{itemize} 
\item Req \#1: Copiar acá el requerimiento.

Verificación y validación:

\begin{itemize}
\item Verificación para confirmar si se cumplió con lo requerido antes de mostrar el sistema al cliente:\\
Detallar 
\item Validación con el cliente para confirmar que está de acuerdo en que se cumplió con lo requerido:\\
Detallar  
\end{itemize}

\end{itemize}

Tener en cuenta que en este contexto se pueden mencionar simulaciones, cálculos, revisión de hojas de datos, consulta con expertos, etc.

\end{consigna}

\section{14. Comunicación del proyecto}
\label{sec:comunicaciones}

\begin{consigna}{red}
El plan de comunicación del proyecto es el siguiente:
\end{consigna}

% Please add the following required packages to your document preamble:
% \usepackage{graphicx}
% \usepackage[table,xcdraw]{xcolor}
% If you use beamer only pass "xcolor=table" option, i.e. \documentclass[xcolor=table]{beamer}
\begin{table}[htpb]
\centering
\resizebox{\textwidth}{!}{%
\begin{tabular}{|c|c|c|c|c|c|}
\hline
\rowcolor[HTML]{C0C0C0} 
\multicolumn{6}{|c|}{\cellcolor[HTML]{C0C0C0}PLAN DE COMUNICACIÓN DEL PROYECTO}           \\ \hline
\rowcolor[HTML]{C0C0C0} 
¿Qué comunicar? & Audiencia & Propósito & Frecuencia & Método de comunicac. & Responsable \\ \hline
                &           &           &            &                      &             \\ \hline
                &           &           &            &                      &             \\ \hline
                &           &           &            &                      &             \\ \hline
                &           &           &            &                      &             \\ \hline
                &           &           &            &                      &             \\ \hline
\end{tabular}%
}
\end{table}

\section{15. Gestión de Compras}
\label{sec:compras}

\begin{consigna}{red}
En caso de tener que comprar elementos o contratar servicios:
a) Explique con qué criterios elegiría a un proveedor.
b) Redacte el Statement of Work correspondiente.
\end{consigna}

\section{16. Seguimiento y control}
\label{sec:seguimiento}

\begin{consigna}{red}
Para cada tarea del proyecto establecer la frecuencia y los indicadores con los se seguirá su avance y quién será el responsable de hacer dicho seguimiento y a quién debe comunicarse la situación (en concordancia con el Plan de Comunicación del proyecto).

El indicador de avance tiene que ser algo medible, mejor incluso si se puede medir en \% de avance. Por ejemplo,se pueden indicar en esta columna cosas como ``cantidad de conexiones ruteadeas'' o ``cantidad de funciones implementadas'', pero no algo genérico y ambiguo como ``\%'', porque el lector no sabe porcentaje de qué cosa.

\end{consigna}

\begin{table}[!htpb]
\centering
\begin{tabularx}{\linewidth}{@{}|X|X|X|X|X|X|@{}}
\hline
\rowcolor[HTML]{C0C0C0} 
\multicolumn{6}{|c|}{\cellcolor[HTML]{C0C0C0}SEGUIMIENTO DE AVANCE}                                                                       \\ \hline
\rowcolor[HTML]{C0C0C0} 
Tarea del WBS & Indicador de avance & Frecuencia de reporte & Resp. de seguimiento & Persona a ser informada & Método de comunic. \\ \hline
 &  &  &  &  &  \\ \hline
 &  &  &  &  &  \\ \hline
 &  &  &  &  &  \\ \hline
 &  &  &  &  &  \\ \hline
 &  &  &  &  &  \\ \hline
\end{tabularx}%
%}
\end{table}

\section{17. Procesos de cierre}    
\label{sec:cierre}

\begin{consigna}{red}
Establecer las pautas de trabajo para realizar una reunión final de evaluación del proyecto, tal que contemple las siguientes actividades:

\begin{itemize}
\item Pautas de trabajo que se seguirán para analizar si se respetó el Plan de Proyecto original:
 - Indicar quién se ocupará de hacer esto y cuál será el procedimiento a aplicar. 
\item Identificación de las técnicas y procedimientos útiles e inútiles que se utilizaron, y los problemas que surgieron y cómo se solucionaron:
 - Indicar quién se ocupará de hacer esto y cuál será el procedimiento para dejar registro.
\item Indicar quién organizará el acto de agradecimiento a todos los interesados, y en especial al equipo de trabajo y colaboradores:
  - Indicar esto y quién financiará los gastos correspondientes.
\end{itemize}

\end{consigna}


\end{document}
